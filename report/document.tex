%%%%%%%%%%%%%%%%%%%%%%%%%%%%%%%%%%%%%%%%%
% University/School Laboratory Report
% LaTeX Template
% Version 3.0 (4/2/13)
%
% This template has been downloaded from:
% http://www.LaTeXTemplates.com
%
% Original author:
% Linux and Unix Users Group at Virginia Tech Wiki
% (https://vtluug.org/wiki/Example_LaTeX_chem_lab_report)
%
% License:
% CC BY-NC-SA 3.0 (http://creativecommons.org/licenses/by-nc-sa/3.0/)
%
%%%%%%%%%%%%%%%%%%%%%%%%%%%%%%%%%%%%%%%%%

%----------------------------------------------------------------------------------------
%	PACKAGES AND DOCUMENT CONFIGURATIONS
%----------------------------------------------------------------------------------------

\documentclass[twocolumn]{article}

\usepackage{mhchem} % Package for chemical equation typesetting
\usepackage{siunitx} % Provides the \SI{}{} command for typesetting SI units
\usepackage{hyperref}
\usepackage{graphicx} % Required for the inclusion of images
\usepackage{tabularx}
\usepackage{float}
\usepackage{algorithm}
\usepackage{algpseudocode}
\usepackage{bm}
\usepackage{multirow}% http://ctan.org/pkg/multirow
\usepackage{hhline}% http://ctan.org/pkg/hhline


\setlength\parindent{0pt} % Removes all indentation from paragraphs

\renewcommand{\labelenumi}{\alph{enumi}.} % Make numbering in the enumerate environment by letter rather than number (e.g. section 6)

%\usepackage{times} % Uncomment to use the Times New Roman font

%----------------------------------------------------------------------------------------
%	DOCUMENT INFORMATION
%----------------------------------------------------------------------------------------

\title{UC Davis STA 242 2015 Spring Assignment 2} % Title
\author{Wenhao \textsc{Wu}, 9987583} % Author name
\date{\today} % Date for the report

\begin{document}
\maketitle % Insert the title, author and date

% If you wish to include an abstract, uncomment the lines below

\section{Data Structure and Algorithm Design}
A \texttt{BMLGrid} instance contains 3 components:
\begin{description}
    \item[\texttt{grid}] A \texttt{r}-by-\texttt{c} integer matrix. If
    \texttt{grid[i,j]==0}, then there is no car on the crossing of $i$-th row
    and $j$-th column; if \texttt{grid[i,j]==1}, there is a red car on that
    grid; if \texttt{grid[i,j]==2}, there is a blue car on that. In our program,
    when \texttt{grid} is indexed, it is treated as a vector (1-D).
    \item[\texttt{blue}] An integer vector contains the 1-D indices of
    all blue cars in \texttt{grid}.
    \item[\texttt{red}] An integer vector contains the 1-D indices of
    all red cars in \texttt{grid}.
\end{description}
We define 2 key functions that returns a vector of 1-D
indices in \texttt{grid}
\begin{description}
    \item[\texttt{idx\_right()}] Given an input vector of 1-D indices in
    \texttt{grid}, return a vector of 1-D indices in \texttt{grid} for grids to
    the \emph{right} of the input grids.
    \item[\texttt{idx\_up()}] Given an input vector of 1-D indices in
    \texttt{grid}, return a vector of 1-D indices in \texttt{grid} for grids to
    the \emph{up} of the input grids.
\end{description}
Upon each step, we use \texttt{idx\_up()}(\texttt{idx\_right()}) to check in
\texttt{grid} whether the grids to the up(right) of the grids represented by
\texttt{blue}(\texttt{red}) is occupied, then update the cars' indices
\texttt{blue}(\texttt{red}) and the grid state \texttt{grid} accordingly.


\section{Simulation Results}
\subsection{Behavior of the BML model}
\subsection{Code Performance}

\section{Build BMLGrid Package}
%\pagebreak
%	BIBLIOGRAPHY
%----------------------------------------------------------------------------------------

%\bibliographystyle{unsrt}
%\bibliography{myrefs}

%----------------------------------------------------------------------------------------


\end{document}